\section{Alit fajar Kurniawan (1174057)}

\subsection{Pengertian}
Geografi adalah ilmu pengetahuan yang mengambarkan segala sesuatu yang ada di permukaan bumi. \hfill\break
Geografi juga selain mempelajari bagian permukaan bumi, tapi juga mempelajari seluruh bagian bumi mulai darti struktur bumi,jenis batuan yang menyusun bumi serta atmosfer yang melindungi bumi \cite{widiani2018efektivitas}. \hfill\break
Segala aktifitas yang terjadi di bumi merupakan bagian dari ilmu Geografi.\hfill\break
\subsection{Sejarah}
Sejarah geografi dimulai sejak manusia mulai berinteraksi dengan lingkunganya, hal ini juga merupakan awal mula dari berkembangnya ilmu pengetahuan tentang geografi.\hfill\break
Pada awalnya geografi hanya membahas atau mendekripsikan gambaran umum tentang fakta-fakta yang menjelaskan keadaan di muka bumi. Pada abad ke-18 yaitu masa geografi klasik, ilmu geografi hanya sebatas menjelaskan dan mengumpulkan informasi tentang lingkungan geografi saja, misalnya: keadaan politik, industri, iklim terutama di kota-kota besar \cite{zuhdi2018sejarah}.\hfill\break
Sejarah geografi terus berjalan dan berkembang. Tepatnya, diabad ke-19 geografi mengalami perkembangan dari segi keilmuannya. Dari yang semula hanya mendeskripsikan saja kemudian berkembang menjadi lebih spesifik yaitu dengan menjelaskan lingkungan geografi secara sistematis.\hfill\break
Pada pertengahan abad ke-19, keilmuan dalam geografi sudah membahas sampai ketingkat membandingkan keadaan, data geografis dan karakteristik antara wilayah yang satu dengan wilayah yang lain di muka bumi. Hal ini kita kenal sebagai “Comparative Geography”.\hfill\break
Perkembangan keilmuwan geografi semakin pesat pasca terjadinya perang dunia ke-II. Yang semula dikembangkan oleh imuwan Amerika dan Inggris yang dikenal sebagai “Comparative Geography” kemudian berkembang menjadi “Global Geography” dimana objek kajiannya semakin luas yaitu meliputi seluruh dunia. Era inilah yang dinamakan sebagai “era geografi modern” \cite{lambert2018geography}.\hfill\break
Dari pembahasan di atas, kita sudah mengetahui kapan sejarah geografi itu dimulai yaitu sejak adanya interaksi antara manusia dengan lingkungannya. Bila seperti itu, maka hakekatnya sejak Nabi Adam as turun ke bumi sebetulnya geografi sudah ada.\hfill\break
Akan tetapi penggalian geografi secara keilmuan sendiri baru dilakukan pertama kali oleh orang-orang Yunani. Dimana pada perkembangan awalnya dilatarbelakangi oleh suatu upaya masyarakat Yunani untuk melepaskan diri dari alam pikiran dan kepercayaan. Dimana kepercayaan tersebut meyakini bahwa dewa-dewa ikut turut campur dalam segala bentuk kejadian di bumi.\hfill\break
Istilah geografi sebenarnya baru digunakan pada tahun 1972 sedangkan sebelumnya lebih menggunakan istilah “ilmu bumi”. Istilah ini pertama kali diperkenalkan oleh seorang ahli filsafat dan astronomi yang bernama Eratosthenes pada 276 194 sebelum masehi.Kemudian, Claudius Ptoleumaeus melakukan peletakan dasar-dasar keilmuan geografi.\hfill\break
Sejarah perkembangan geografi terus berlanjut. Immanuel Kant mengembangkan geografi modern kemudian Karl Ritter juga mengembangkan geografi sosial.\hfill\break
Selain itu ada tokoh-tokoh lain yang ikut andil dalam mengembangkan geografi yaitu Alexander von Humbolt sebagai peletak dasar geografi fisika modern dan sebagainya.\hfill\break

\begin{figure}[H]
	\includegraphics[width=4cm]{figures/1174057/sejarah.jpg}
	\centering
	\caption{Sejarah}
\end{figure}

\subsection{Link}
\href{https://www.youtube.com/watch?v=XdUSIKaI3zc}{Nonton Video aku di Youtube}
\subsection{Plagiarism}
\begin{figure}[H]
	\includegraphics[width=4cm]{figures/1174057/plagiarisme.png}
	\centering
	\caption{Plagiarism}
\end{figure}